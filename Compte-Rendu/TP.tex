\documentclass{article}
\usepackage[french]{babel}
\usepackage[latin1]{inputenc}
\usepackage{array}
\usepackage{amsmath}
\usepackage{tikz}
\usetikzlibrary{arrows}
\usepackage{mathrsfs}
\usepackage{float}
\usepackage{caption}
\usepackage[ruled,vlined,french,onelanguage]{algorithm2e}
\title{TP M�thode approch�es UMIN215}
\author{Bruno Y., Chl�� T., Julien D. et R�mi F.}


\begin{document}
\maketitle

\subsection*{Partie Th�orique}

\subsubsection*{Exercice 1 - Sur le probl�me de la couverture sommet minimale : trois approches diff�rentes}

\subsubsection*{Exercice 2 - Sur le probl�me du couplage maximum de poids minimum : un d�but d'�tude poly�drale sur le probl�me de couplage}

\subsubsection*{Exercice 3 - Sur le probl�me de la coupe maximum}

\subsubsection*{Exercice 4 - Sur le probl�me de Partition}

\subsubsection*{Exercice 5 - Sur le probl�me du sac � dos simple}

\subsubsection*{Exercice 6 - Programmation dynamique}

\subsubsection*{Exercice 7 - Sur le produit matriciel}

\subsubsection*{Exercice 8 - R�solution num�rique}

\subsubsection*{Exercice 9 - Seuil d'approximation pour le probl�me Bin Packing}

\subsubsection*{Exercice 10 - Seuil d'approximation pour le probl�me de la coloration de sommets (reps. d'ar�tes)}

\subsubsection*{Exercice 11 - Comparaisons branch and bound and branch and cut}

\subsection*{Partie Pratique}

\subsubsection*{2.1 Programmation dynamique}

\subsubsection*{2.2 Branch and Bound}

\subsubsection*{2.3 Comparaisons entre un algorithme de complexit� exponentielle et un FP-TAS}

\end{document}
