\documentclass{article}
\usepackage[french]{babel}
\usepackage[latin1]{inputenc}
\usepackage{array}
\usepackage{amsmath}
\usepackage{tikz}
\usetikzlibrary{arrows}
\usepackage{mathrsfs}
\usepackage{float}
\usepackage{pgfplots}
\usepackage{caption}
\usepackage[ruled,vlined,french,onelanguage]{algorithm2e}
\title{TP M�thode approch�es UMIN215}
\author{Bruno Y., Chl�� T., Julien D. et R�mi F.}

\begin{document}
\maketitle

\subsection*{Partie Th�orique}

\subsubsection*{Exercice 1 - Sur le probl�me de la couverture sommet minimale : trois approches diff�rentes}

\subsubsection*{Exercice 2 - Sur le probl�me du couplage maximum de poids minimum : un d�but d'�tude poly�drale sur le probl�me de couplage}

\subsubsection*{Exercice 3 - Sur le probl�me de la coupe maximum}

\subsubsection*{Exercice 4 - Sur le probl�me de Partition}

\subsubsection*{Exercice 5 - Sur le probl�me du sac � dos simple}

\subsubsection*{Exercice 6 - Programmation dynamique}

\subsubsection*{Exercice 7 - Sur le produit matriciel}

\subsubsection*{Exercice 8 - R�solution num�rique}

\subsubsection*{Exercice 9 - Seuil d'approximation pour le probl�me Bin Packing}

\subsubsection*{Exercice 10 - Seuil d'approximation pour le probl�me de la coloration de sommets (reps. d'ar�tes)}

\newpage
\subsubsection*{Exercice 11 - Comparaisons branch and bound and branch and cut}

\begin{enumerate}
\item On peut tracer les droites correspondantes aux contraintes de $PL_{0}$ :
\begin{itemize}
\item {$y_{1}=5-\frac{3}{2}x$}
\item {$y_{2}=\frac{17}{2}-\frac{2}{5}x$}
\end{itemize}

\begin{figure}[h]
\centering
\begin{tikzpicture}[>=stealth]
    \begin{axis}[
        xmin=0,xmax=10,
        ymin=0,ymax=6,
        axis x line=middle,
        axis y line=middle,
        axis line style=->,
        xlabel={$x$},
        ylabel={$y$},
        ]
        \addplot[no marks,black,-] expression[domain=0:10/3,samples=100]{5-(3/2)*x} 
                    node[pos=0.65,anchor=south west]{$y_{1}=5-\frac{3}{2}x$}; 
        \addplot[no marks,black,-] expression[domain=0:17/2,samples=100]{17/5-(2/5)*x} 
                    node[pos=0.65,anchor=south west]{$y_{2}=\frac{17}{5}-\frac{2}{5}x$}; 
    \end{axis}
\end{tikzpicture}
   \caption{Repr�sentation des �quations de $PL_{0}$.}
\end{figure}

Le polytope associ� aux �quations de $PL_{0}$ est donc celui form� par les points : $P_{0}(0,0)$, $P_{1}(0,\frac{17}{5})$, $P_{2}(\frac{16}{11}, \frac{31}{11})$ et $P_{3}(\frac{10}{3},0)$

\item On peut tracer la fonction objective :
\begin{itemize}
\item $y=-2x+k, k \in \mathbf{R}$
\end{itemize}

\begin{tikzpicture}[>=stealth]
    \begin{axis}[
        xmin=0,xmax=10,
        ymin=0,ymax=6,
        axis x line=middle,
        axis y line=middle,
        axis line style=->,
        xlabel={$x$},
        ylabel={$y$},
        ]
        \addplot[no marks,black,-] expression[domain=0:10/3,samples=100]{5-(3/2)*x} 
                    node[pos=0.65,anchor=south west]{$y_{1}=5-\frac{3}{2}x$}; 
        \addplot[no marks,blue,-] expression[domain=0:17/2,samples=100]{17/5-(2/5)*x} 
                    node[pos=0.65,anchor=south west]{$y_{2}=\frac{17}{5}-\frac{2}{5}x$}; 
                    
	 \addplot[no marks,red,-] expression[domain=0:17/2,samples=100]{-2*x+20/3} 
                    node[pos=0.65,anchor=south west]{Fonction objective}; 
    \end{axis}
\end{tikzpicture}

La solution optimale pour $PL_{0}$ est donc $x_{1}=\frac{10}{3}$ et $x_{2}={0}$ avec $z=\frac{20}{3}$.

\item On commence par reprendre le programme lin�aire donn� :
\begin{displaymath}
PL_{0}\begin{cases} 
\text{max } z(x_{1},x_{2})=2x_{1}+x_{2} \\
2x_{1}+5x_{2} \leq 17 \\
3x_{1}+2x_{2} \leq 10 \\
x_{1}, x_{2} \geq 0
\end{cases}
\end{displaymath}
On ajoute les variables d'�carts $x_{3}$ et $x_{4}$ pour obtenir $PL_{1}$ :

\begin{displaymath}
PL_{1}\begin{cases} 
\text{max } z(x_{1},x_{2})=2x_{1}+x_{2} \\
2x_{1}+5x_{2} + x_{3} = 17 \\
3x_{1}+2x_{2} + x_{4} = 10 \\
x_{1}, x_{2} \geq 0
\end{cases}
\end{displaymath}


\begin{tabular}{|p{0.75cm}|p{0.75cm}|p{0.75cm}|p{0.75cm}|p{0.75cm}|p{0.75cm}|p{0.75cm}|}
   \hline
  \multicolumn{3}{|c|}{} & 2 & 1 & 0 & 0\\
     \hline
   & && $x_{1}$&$x_{2}$ &$x_{3}$ &$x_{4}$\\
   \hline
   0&$x_{3}$ &17& 2&5 &1 &0\\
   \hline
   0&$x_{4}$ & 10 &3 & 2&0 &1\\
   \hline
      & $z$& 0&-2 &-1 &0 &0\\
      \hline
\end{tabular}

\begin{tabular}{|p{0.75cm}|p{0.75cm}|p{0.75cm}|p{0.75cm}|p{0.75cm}|p{0.75cm}|p{0.75cm}|}
   \hline
  \multicolumn{3}{|c|}{} & 2 & 1 & 0 & 0\\
   \hline
   & && $x_{1}$&$x_{2}$ &$x_{3}$ &$x_{4}$\\
   \hline
   0&$x_{3}$ &$\frac{31}{3}$& 0&$\frac{11}{3} $&1 &$\frac{-2}{3}$\\
   \hline
   2&$x_{1}$ & $\frac{10}{3}$ &1 & $\frac{2}{3}$&0 & $\frac{1}{3}$\\
   \hline
      & $z$& $\frac{20}{3}$&0 &$\frac{1}{3}$ &0 &$\frac{2}{3}$\\
      \hline
\end{tabular}

Puisque que l'on a $\frac{31}{3} = \frac{11}{3} x_{2} + x_{3} - \frac{2}{3} x_{4} $, on peut d�duire que : $ \frac{11}{3} x_{2} - \frac{2}{3} x_{4} \geq \frac{1}{3} $. \\
Puisque l'on a $\frac{10}{3} = x_{1} + \frac{2}{3} x_{2} + \frac{1}{3}x_{4}$ , on peut donc d�duire que $x_{4}=10 - 3x_{1} - 2x_{2}$. \\
On obtient finalement la contrainte : $2x_{1} + 5x_{2} \geq 7$. 

Le tableau final donne :

\begin{tabular}{|p{0.75cm}|p{0.75cm}|p{0.75cm}|p{0.75cm}|p{0.75cm}|p{0.75cm}|p{0.75cm}|p{0.75cm}|}
   \hline
  \multicolumn{3}{|c|}{} & 2 & 1 & 0 & 0\\
   \hline
   & && $x_{1}$&$x_{2}$ &$x_{3}$ &$x_{4}$ & $x_{5}$\\
   \hline
   0&$x_{3}$ &$\frac{31}{3}$& 0&$\frac{11}{3} $&1 &$\frac{-2}{3}$\\
   \hline
   2&$x_{1}$ & $\frac{10}{3}$ &1 & $\frac{2}{3}$&0 & $\frac{1}{3}$\\
   \hline
      & $z$& $\frac{20}{3}$&0 &$\frac{1}{3}$ &0 &$\frac{2}{3}$\\
      \hline
\end{tabular}


\end{enumerate}
\subsection*{Partie Pratique}

\subsubsection*{2.1 Programmation dynamique}

\subsubsection*{2.2 Branch and Bound}

\subsubsection*{2.3 Comparaisons entre un algorithme de complexit� exponentielle et un FP-TAS}

\end{document}
